% Created 2013-09-22 Sun 22:27
\documentclass[11pt]{article}
\usepackage[utf8]{inputenc}
\usepackage[T1]{fontenc}
\usepackage{fixltx2e}
\usepackage{graphicx}
\usepackage{longtable}
\usepackage{float}
\usepackage{wrapfig}
\usepackage[normalem]{ulem}
\usepackage{textcomp}
\usepackage{marvosym}
\usepackage{wasysym}
\usepackage{latexsym}
\usepackage{amssymb}
\usepackage{amstext}
\usepackage{hyperref}
\tolerance=1000
\usepackage{savesym}
\savesymbol{iint}
\savesymbol{iiint}
\usepackage{amsmath}
\usepackage{tikz}
\usepackage{tikz-cd}
\usetikzlibrary{matrix,arrows,positioning,scopes,chains}
\tikzset{node distance=2cm, auto}
\usepackage{framed}
\usepackage[framed]{ntheorem}
\newframedtheorem{myrule}{Rule}[section]
\newframedtheorem{mydefinition}{Definition}[section]
\author{The Team of Fu}
\date{\today}
\title{Reversible Arithmetic on Collections}
\hypersetup{
  pdfkeywords={},
  pdfsubject={},
  pdfcreator={Emacs 24.3.1 (Org mode 8.0)}}
\begin{document}

\maketitle
\tableofcontents


\begin{description}
\item[{Remark}] This is a literate program.
\footnote{\url{http://en.wikipedia.org/wiki/Literate_programming}.}
Source code \emph{and} PDF documentation spring
from the same, plain-text source files.
\end{description}

\section{Introduction}
\label{sec-1}

We often encounter data records or rows as hash-maps, lists, vectors
(also called \emph{arrays}). In our financial calculations, we often want
to add up a collection of such things, where adding two rows means
adding the corresponding elements and creating a new virtual row from
the result. We also want to \emph{un-add} so we can undo a mistake, roll
back a provisional result, perform a backfill or allocation: in short,
get back the original inputs. This paper presents a library supporting
reversible on a large class of collections in
Clojure.\footnote{\url{http://clojure.org}}
\section{Mathematical Background}
\label{sec-2}

Think of computer lists and vectors as \emph{mathematical vectors} familiar
from linear algebra:\footnote{\url{http://en.wikipedia.org/wiki/Linear_algebra}}
ordered sequences of numerical \emph{components} or \emph{elements}. Think of
hash-maps, which are equivalent to \emph{objects} in object-oriented
programming,\footnote{\url{http://en.wikipedia.org/wiki/Object-oriented_programming}}
as sparse vectors\footnote{\url{http://en.wikipedia.org/wiki/Sparse_vector}} of
\emph{named} elements.

Mathematically, arithmetic on vectors is straightforward: to add
them, just add the corresponding elements, first-with-first,
second-with-second, and so on.  Here's an example in two dimensions:
$$[1, 2] + [3, 4] = [4, 6]$$

Clojure's \emph{map} function does mathematical vector addition straight
out of the box on Clojure vectors and lists.  (We don't need to write
the commas, but we can if we want -- they're just whitespace in
Clojure):
\begin{verbatim}
(map + [1 2] [3 4])
\end{verbatim}

\begin{verbatim}
==> [4 6]
\end{verbatim}

With Clojure hash-maps, add corresponding elements via \emph{merge-with}:
\begin{verbatim}
(merge-with + {:x 1, :y 2} {:x 3, :y 4})
\end{verbatim}

\begin{verbatim}
==> {:x 4, :y 6}
\end{verbatim}

The same idea works in any number of dimensions and with any kind of
elements that can be added (any \emph{mathematical
field}:\footnote{\url{http://en.wikipedia.org/wiki/Field_(mathematics)}}
integers, complex numbers, quaternions -- many more.

Now, suppose you want to \emph{un-add} the result, \verb|[4 6]|? There is
no unique answer.  All the following are mathematically correct:
\begin{align*}
[-1, 2] + [5, 4] &= [4, 6] \\
[ 0, 2] + [4, 4] &= [4, 6] \\
[ 1, 2] + [3, 4] &= [4, 6] \\
[ 2, 2] + [2, 4] &= [4, 6] \\
[ 3, 2] + [1, 4] &= [4, 6] \\
\end{align*}
and a large infinity of more answers.
\section{A Protocol for Reversible Arithmetic}
\label{sec-3}

Let's define a protocol for \emph{reversible arithmetic in vector spaces}
that captures the desired functionality.  We want a \emph{protocol} --
Clojure's word for
\emph{interface},\footnote{\url{http://en.wikipedia.org/wiki/Interface_(computing)}}
because we want several implementations with the same reversible
arithmetic: one implementation for vectors and lists, another
implementation for hash-maps.  \emph{Protocols} let us ignore inessential
differences: the protocol for reversible arithmetic is the same for
all compatible collection
types.\footnote{including streams over time! Don't forget Rx and SRS.}

Name our objects of interest \emph{algebraic vectors} to distinguish them
from Clojure's existing \emph{vector} type. Borrowing an idiom from C\# and
.NET, name our protocol with an initial \emph{I} and with camelback
casing.\footnote{\url{http://en.wikipedia.org/wiki/CamelCase}} Don't misread
\emph{IReversibleAlgebraicVector} as ``irreversible algebraic vector;''
rather read it as ``I Reversible Algebraic Vector'', i.e., ``Interface
to Reversible Algebraic Vector,'' where the ``I'' abbreviates
``Interface.''

We want to add, subtract, and scale our reversible vectors, just as we
can do with mathematical vectors.  \emph{Add} should be multiary, because
that's intuitive.  Multiary functions are written with ampersands
before all the optional parameters, which Clojure will package in a
sequence. Unfortunately, protocol functions don't support ``rest''
arguments,\footnote{\url{http://bit.ly/18kecbJ}} so we'll go with binary \emph{add}
for now, and built up multiarity through functional programming mojo.

\emph{Sub} should be binary, because multiary sub is ambiguous.

Include inner product, because it is likely to be useful.

Though we don't have immediate scenarios for subtraction, scaling, and
inner product, the mathematics tells us they're fundamental. Putting
them in our design \emph{now} affords two benefits:
\begin{enumerate}
\item when the need arises, we won't have to change the code
\item their existence in the library may inspire usage scenarios

\item[{Remark}] The choice to include operations in a library in the absense
of scenarios is
philosophical,\footnote{\url{http://en.wikipedia.org/wiki/Design_philosophy}}
perhaps more akin to \emph{Action-Centric} design or \emph{proactive}
design as opposed to \emph{Rationalist} or \emph{minimalist}
design. The former philosophy promotes early inclusion of
facilities likely to be useful or inspirational, whereas the
latter philosophy demands ruthless rejection of facilities
not known to be needed. Both require removing facilities
\emph{known to be not needed}, of course. The former philosophy
relies on intuition, taste, judgment, and experience; and
the latter philosophy embraces ignorance of the future as a
design principle. We thus prefer the former.
\end{enumerate}

Finally, we need \emph{undo} and \emph{redo}, the differentiating features of
reversible algebraic vectors. Unlike object-oriented programming,
there is no implicit \emph{this} instance.  Here is our protocol design:

\begin{figure}[H]
\label{reversible-algebraic-vector-protocol}
\begin{verbatim}
(defprotocol IReversibleAlgebraicVector
  ;; binary operators
  (add   [this that])
  (sub   [this that])
  (inner [this that])
  ;; unary operators
  (scale [this scalar])
  ;; reverse any operation
  (undo [this])
  (redo [this])
)
\end{verbatim}
\end{figure}
\section{Implementing the Protocol}
\label{sec-4}

\subsection{Defining \emph{r-vectors} and \emph{a-vectors}}
\label{sec-4-1}

What things represent algebraic vectors?  Things we can operate on
with \emph{map} or \emph{merge-with} to perform basic vector-space operations.
Therefore, they must be Clojure vectors, lists, or hash-maps.

The higher-level case wraps reversing information in a hash-map along
with base-case algebraic vector data. The base data will belong to
the \emph{\mbox{:a-vector}} key, by convention.


\begin{mydefinition}[Algebraic Vector (a-vector)]
   An \textbf{a-vector} is a Clojure vector, a list, or a hash-map that does not
   contain an \mbox{\texttt{:a-vector}} attribute.
\end{mydefinition}

\begin{mydefinition}[Reversible Algebraic Vector (r-vector)]
   A \textbf{reversible algebraic vector} or \textbf{r-vector} is a
   hash-map containing an \texttt{:a-vector} attribute. The value of
   that attribute must be an a-vector.
\end{mydefinition}
\subsection{Checking the Definition}
\label{sec-4-2}

Here is a type-checking function for \emph{a-vector}. This function is
private to the namespace (that's what the `-' in \emph{defn-} means).  It
takes a single parameter named \emph{that}. It promotes \emph{fluent} or
function-chaining style by being, semantically, the identity
function. It either returns its input or throws an exception if
something is wrong.

\begin{figure}[H]
\label{check-a-vector}
\begin{verbatim}
(defn- check-a-vector [that]
  (if (or (list? that)
          (vector? that)
          (and (map? that) (not (contains? that :a-vector))))
      that ; ok -- otherwise:
      (throw (IllegalArgumentException.
        (str "; This type can't hold a-vector data: "
             (type that))))))
\end{verbatim}
\end{figure}
\subsection{Fetching \emph{a-vector} Data}
\label{sec-4-3}

We need a way to get a-vector data out of any r-vector.

\begin{figure}[H]
\label{get-a-vector}
\begin{verbatim}
(defn get-a-vector [that]
  (if (not (map? that))
      (throw (IllegalArgumentException. (str that)))
      (check-a-vector (:a-vector that))))
\end{verbatim}
\end{figure}
\subsection{Unit-Testing \emph{get-a-vector}}
\label{sec-4-4}

We require \emph{IllegalArgumentExceptions} for inputs that are not
a-vectors and for r-vectors that contain r-vectors: our design does
not nest r-vectors.

Let's make test sets for data that should be accepted and rejected
immediately.  Creating new tests is as easy as adding new instances
to these sets.  Include some types that may not be acceptable for
arithmetic; we are just testing structure here.

\begin{figure}[H]
\label{test-data-sets}
\begin{verbatim}
(def ^:private atoms
  '(42 42.0 42.0M 42N 'a :a "a" \a
    #inst "2012Z"
    #{} #{0} nil true false))

(def ^:private vectors
  (concat [[]] (map vector atoms)))

(def ^:private lists
  (concat [()] (map list atoms)))

(def ^:private maps
  (concat [{}] (map (fn [a] {:a a}) atoms)))

(def ^:private a-vectors
  (concat (map (fn [a] {:a-vector a}) vectors)
          (map (fn [a] {:a-vector a}) lists)
          (map (fn [a] {:a-vector a}) maps)))

(def ^:private good-ish-test-collection
  a-vectors)

(def ^:private bad-ish-test-collection
  (concat maps
          (map (fn [a] {:a-vector a}) atoms)
          (map (fn [a] {:a-vector a}) a-vectors)))
\end{verbatim}
\end{figure}

We cannot just \emph{map} or iterate \emph{get-a-vector} over bad inputs
because Clojure evaluates arguments
eagerly.\footnote{\url{http://en.wikipedia.org/wiki/Evaluation_strategy#Applicative_order}}
The first exception will terminate the entire \emph{map} operation, but we
want to test that they all throw exceptions.

One way to defeat eager evaluation is with a higher-order function.
\footnote{another, more complicated way is with a \emph{macro}, which rewrites
   expressions at compile time. Macrros should be avoided when
   functional alternatives exist because they are hard to develop and
   debug.}  Pass \emph{get-a-vector} as a function to another function that
wraps it in a \emph{try} that converts an exception into a string.
Collect all bad-ish strings into a hash-set and test that the
hash-set contain just the string
``\emph{java.lang.IllegalArgumentException}.'' For the \emph{good-ish}
test set, map the values into a sequence that should match the inputs
in order.

\begin{figure}[H]
\label{get-a-vector-test}
\begin{verbatim}
(defn- exception-to-name [fun expr]
  (try (fun expr)
       (catch Exception e (re-find #"[^:;,]+" (str e)))))

(defn- value-seq [fun exprs]
  (map (fn [x] (exception-to-name fun x)) exprs))

(defn- value-set [fun exprs]
  (set (value-seq fun exprs)))

(deftest get-a-vector-test
  (testing "get-a-vector"
    ;; Negative tests
    (is (= #{"java.lang.IllegalArgumentException"}
           (value-set get-a-vector bad-ish-test-collection)))
    ;; Positive tests
    (is (= (map :a-vector good-ish-test-collection)
           (value-seq get-a-vector good-ish-test-collection))) ) )
\end{verbatim}
\end{figure}
\subsection{Dispatching Operations by Collection Type}
\label{sec-4-5}

To implement the protocol, we need multimethods that dispatch on the
collection types of the a-vectors. Lists and Clojure vectors should
be treated the same: as sequences. Let's call them
\emph{seq-ish}. Hash-maps should be treated as \emph{map-ish}. All other types
are illegal.

\begin{figure}[H]
\label{one-type}
\begin{verbatim}
(defn one-type [a]
  (cond
    (or (vector? a) (list? a)) 'seq-ish
    (map? a)                   'map-ish
    :default (throw (IllegalArgumentException. (str a)))))
\end{verbatim}
\end{figure}

To dispatch on collection type, test the types of all inputs. Here,
again, we see ampersands before a parameter representing a sequence
of all optional arguments.

\begin{figure}[H]
\label{add-a-vectors}
\begin{verbatim}
(defn- all-types [& exprs] (set (map one-type exprs)))
(defmulti  add-a-vectors all-types)
(defmethod add-a-vectors #{'seq-ish} [& those]
  (apply map + those))
(defmethod add-a-vectors #{'map-ish} [& those]
  (apply merge-with + those))
(defmethod add-a-vectors :default    [& those]
  (throw (IllegalArgumentException.
    (str "; Illegal type combination: " (map type those)))))
\end{verbatim}
\end{figure}

At this point, it is worth noting that \emph{static typing} -- types
tested by a compiler -- would save us the work of writing run-time
type tests, but at the expense of the build-time and run-time
complexity of introducing another language into our data-processing
pipeline. This complexity tradeoff -- coding versus building and
running -- is a judgment call.  We stick with dynamic type-checking,
the only option available in Clojure, for now.

Our \emph{add-a-vectors} function is loose: it will add one or more
a-vectors, where our protocol will only accept two or more. This is
fine: it only means that we unit test a few more cases for
\emph{add-a-vectors} than for our protocol.

Regarding the underlying arithmetic: if we attempt to add values that
cannot be added via the $+$ operator, we do not interfere with the
underlying exceptions that Clojure and Java may throw. Therefore, we
do not need to test such cases here.

\begin{figure}[H]
\label{add-a-vectors-test}
\begin{verbatim}
(deftest add-a-vectors-test
  (testing "add-a-vectors")
  (is (= #{"java.lang.IllegalArgumentException"}
         (value-set add-a-vectors atoms)))
  (are [expr] (thrown? java.lang.IllegalArgumentException expr)
       (add-a-vectors '()  {})
       (add-a-vectors  []  {})
       (add-a-vectors  {} '())
       (add-a-vectors  {}  [])
       (add-a-vectors))
  (are [x y] (= x y)
    (add-a-vectors [])    []
    (add-a-vectors [1])   [1]
    (add-a-vectors [1 1]) [1 1]

    (add-a-vectors '())    '()
    (add-a-vectors '(1))   '(1)
    (add-a-vectors '(1 1)) '(1 1)

    (add-a-vectors [1]   [2])   [3]
    (add-a-vectors [1 2] [3 4]) [4 6]

    (add-a-vectors '(1)   '(2))   '(3)
    (add-a-vectors '(1 2) '(3 4)) '(4 6)

    (add-a-vectors '(1)   [2])   [3]
    (add-a-vectors '(1 2) [3 4]) [4 6]

    (add-a-vectors [1]   '(2))   [3]
    (add-a-vectors [1 2] '(3 4)) [4 6]

    (add-a-vectors [1]   [2])   '(3)
    (add-a-vectors [1 2] [3 4]) '(4 6)

    (add-a-vectors '(1)   [2])   '(3)
    (add-a-vectors '(1 2) [3 4]) '(4 6)

    (add-a-vectors [1]   '(2))   '(3)
    (add-a-vectors [1 2] '(3 4)) '(4 6)

    (add-a-vectors [1]   [2 3])   [3]
    (add-a-vectors [1 2] [3 4 5]) [4 6]

    (add-a-vectors {})          {}

    (add-a-vectors {:a 1})      {:a 1}
    (add-a-vectors {:a 1 :b 2}) {:a 1 :b 2}

    (add-a-vectors {:a 1} {})      {:a 1}
    (add-a-vectors {:a 1 :b 2} {}) {:a 1 :b 2}

    (add-a-vectors {} {:a 1})      {:a 1}
    (add-a-vectors {} {:a 1 :b 2}) {:a 1 :b 2}

    (add-a-vectors {} {:a 1})      {:a 1}
    (add-a-vectors {} {:a 1 :b 2}) {:a 1 :b 2}

    (add-a-vectors {} {:a 1} {})      {:a 1}
    (add-a-vectors {} {:a 1 :b 2} {}) {:a 1 :b 2}

    (add-a-vectors {:a 1} {:a 2})           {:a 3}
    (add-a-vectors {:a 1 :b 2} {:a 3 :b 4}) {:a 4 :b 6}

    (add-a-vectors {:a 1} {:b 2})           {:a 1 :b 2}
    (add-a-vectors {:a 1 :b 2} {:a 3 :c 4}) {:a 4 :b 2 :c 4}
  ) )
\end{verbatim}
\end{figure}
\subsection{The ReversibleVector Type}
\label{sec-4-6}

We now have enough to implement the \emph{add} method of the protocol.

\begin{figure}[H]
\label{reversible-algebraic-vector-on-vector}
\begin{verbatim}
(defrecord ReversibleVector [my-r-vector]
  IReversibleAlgebraicVector
  (add   [this that]
         (let [priors [(get-a-vector (.my-r-vector this))
                       (get-a-vector (.my-r-vector that))]]
           (->ReversibleVector
             {:priors    priors
              :operation 'add,
              :a-vector  (apply add-a-vectors priors)})))
  (sub   [this that] nil)
  (inner [this that] nil)
  (scale [this scalar] nil)
  (undo  [this] nil)
  (redo  [this] nil))
\end{verbatim}
\end{figure}
\section{Unit-Tests}
\label{sec-5}

\begin{figure}[H]
\label{test-namespace}
\begin{verbatim}
(ns ex1.core-test
  (:require [clojure.test :refer :all]
            [ex1.core     :refer :all]))
\end{verbatim}
\end{figure}
\section{REPLing}
\label{sec-6}
\label{sec:emacs-repl}
To run the REPL for interactive programming and testing in org-mode,
take the following steps:
\begin{enumerate}
\item Set up emacs and nRepl (TODO: explain; automate)
\item Edit your init.el file as follows (TODO: details)
\item Start nRepl while visiting the actual |project-clj| file.
\item Run code in the org-mode buffer with \verb|C-c C-c|; results of
evaluation are placed right in the buffer for inspection; they are
not copied out to the PDF file.
\end{enumerate}
% Emacs 24.3.1 (Org mode 8.0)
\end{document}
