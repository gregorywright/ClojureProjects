% Created 2013-09-22 Sun 08:09
\documentclass[11pt]{article}
\usepackage[utf8]{inputenc}
\usepackage[T1]{fontenc}
\usepackage{fixltx2e}
\usepackage{graphicx}
\usepackage{longtable}
\usepackage{float}
\usepackage{wrapfig}
\usepackage[normalem]{ulem}
\usepackage{textcomp}
\usepackage{marvosym}
\usepackage{wasysym}
\usepackage{latexsym}
\usepackage{amssymb}
\usepackage{amstext}
\usepackage{hyperref}
\tolerance=1000
\usepackage{savesym}
\savesymbol{iint}
\savesymbol{iiint}
\usepackage{amsmath}
\usepackage{tikz}
\usepackage{tikz-cd}
\usetikzlibrary{matrix,arrows,positioning,scopes,chains}
\tikzset{node distance=2cm, auto}
\usepackage{framed}
\usepackage[framed]{ntheorem}
\newframedtheorem{myrule}{Rule}[section]
\newframedtheorem{mydefinition}{Definition}[section]
\author{The Team of Fu}
\date{\today}
\title{Reversible Arithmetic on Collections}
\hypersetup{
  pdfkeywords={},
  pdfsubject={},
  pdfcreator={Emacs 24.3.1 (Org mode 8.0)}}
\begin{document}

\maketitle
\tableofcontents


\begin{description}
\item[{Remark}] This is a literate program.
\footnote{\url{http://en.wikipedia.org/wiki/Literate_programming}.}
Source code \emph{and} PDF documentation spring
from the same, plain-text source files.
\end{description}

\section{Introduction}
\label{sec-1}

We often encounter data records or rows as hash-maps, lists, vectors
(also called \emph{arrays}). In our financial calculations, we often want
to add up a collection of such things, where adding two rows means
adding the corresponding elements and creating a new virtual row from
the result. We also want to \emph{un-add} so we can undo a mistake, roll
back a provisional result, perform a backfill or allocation: in short,
get back the original inputs. This paper presents a library supporting
reversible on a large class of collections in
Clojure.\footnote{\url{http://clojure.org}}
\section{Mathematical Background}
\label{sec-2}

Think of computer lists and vectors as \emph{mathematical vectors} familiar
from linear algebra:\footnote{\url{http://en.wikipedia.org/wiki/Linear_algebra}}
ordered sequences of numerical \emph{components} or \emph{elements}. Think of
hash-maps, which are equivalent to \emph{objects} in object-oriented
programming,\footnote{\url{http://en.wikipedia.org/wiki/Object-oriented_programming}}
as sparse vectors\footnote{\url{http://en.wikipedia.org/wiki/Sparse_vector}} of
\emph{named} elements.

Mathematically, arithmetic on vectors is straightforward: to add
vectors, just add the corresponding elements, first-with-first,
second-with-second, and so on.  Here's an example in two dimensions:
$$[1, 2] + [3, 4] = [4, 6]$$

Clojure's \emph{map} function does mathematical vector addition straight
out of the box on Clojure vectors and lists.  (We don't need to write
the commas, but we can if we want -- they're just whitespace):
\begin{verbatim}
(map + [1 2] [3 4])
\end{verbatim}

\begin{verbatim}
==> [4 6]
\end{verbatim}

With Clojure hash-maps, we can add corresponding elements via
\emph{merge-with}
\begin{verbatim}
(merge-with + {:x 1, :y 2} {:x 3, :y 4})
\end{verbatim}

\begin{verbatim}
==> {:x 4, :y 6}
\end{verbatim}

The same idea works in any number of dimensions and with any kind of
elements that can be added (any \emph{mathematical
field}:\footnote{\url{http://en.wikipedia.org/wiki/Field_(mathematics)}}
integers, complex numbers, quaternions, many more.

Now, suppose you want to \emph{un-add} the result, \verb|[4 6]|? There is
no unique answer.  All the following are mathematically correct:
\begin{align*}
[-1, 2] + [5, 4] &= [4, 6] \\
[ 0, 2] + [4, 4] &= [4, 6] \\
[ 1, 2] + [3, 4] &= [4, 6] \\
[ 2, 2] + [2, 4] &= [4, 6] \\
[ 3, 2] + [1, 4] &= [4, 6] \\
\end{align*}
and a large infinity of more answers.
\section{A Protocol for Reversible Arithmetic}
\label{sec-3}

Let's define a protocol for \emph{reversible arithmetic in vector spaces}
that captures the desired functionality.  We want a \emph{protocol} --
Clojure's word for
\emph{interface},\footnote{\url{http://en.wikipedia.org/wiki/Interface_(computing)}}
because we want several implementations with the same reversible
arithmetic: one implementation for vectors and lists, another
implementation for hash-maps.  Protocols let us ignore inessential
differences: the protocol for reversible arithmetic on
data-rows-as-mathematical-vectors should be the same for all
collection types.\footnote{including streams over time! Don't forget Rx and
  SRS.}

Name our objects of interest \emph{algebraic vectors} to distinguish them
from Clojure's existing \emph{vector} type. Borrowing an idiom from C\# and
.NET, name our protocol with an initial \emph{I} and with camelback
casing.\footnote{\url{http://en.wikipedia.org/wiki/CamelCase}} Don't misread
\emph{IReversibleAlgebraicVector} as ``irreversible algebraic vector;''
rather read it as ``Interface to Reversible Algebraic Vector,'' where
the ``I'' abbreviates ``Interface.''

We want to add, subtract, and scale our reversible vectors, just as we
can do with mathematical vectors. We include inner product, since it
is likely to be useful. Though we don't have an immediate scenario for
subtraction and inner product, the mathematics tells us they're
fundamental. Putting them in our design \emph{now} affords two benefits:
\begin{enumerate}
\item when the need arises, we won't have to change the code
\item their existence in the library may inspire usage scenarios
\end{enumerate}



\begin{description}
\item[{Remark}] The choice to include operations in a library in the absense
of scenarios is a philosophical
choice,\footnote{\url{http://en.wikipedia.org/wiki/Design_philosophy}}
perhaps more akin to \emph{Action-Centric design} or \emph{proactive}
design as opposed to \emph{Rational} or \emph{minimalist} design. The
former philosophy promotes early inclusion of facilities
likely to be useful, where as the latter demands ruthless
reduction of facilities not known to be needed. The former
is based on intuition, judgment, and experience, and the
latter is based on ignorance of the future. We thus prefer
the former.
\end{description}



Finally, we need undo and redo, the differentiating features of
reversible algebraic vectors. Here is our protocol design:

\begin{figure}[H]
\label{reversible-algebraic-vector-protocol}
\begin{verbatim}
(defprotocol IReversibleAlgebraicVector
  ;; binary operators
  (add   [a b])
  (sub   [a b])
  (inner [a b])
  ;; unary operators
  (scale [a scalar])
  ;; reverse any operation
  (undo [a])
  (redo [a])
)
\end{verbatim}
\end{figure}

\subsection{Implementing the Protocol for Vectors and Lists}
\label{sec-3-1}

As a first cut, let us wrap algebraic vectors in hash-maps that contain
enough information to reverse any computation.

First, define the base case: collections that hold data that can be
treated as ordinary, non-reversible, algebraic vectors.  What kinds of
things can hold ordinary vector data?  They must be things we can
operate on with \emph{map} or \emph{merge-with} to perform the basic, vector-space
operations.  Therefore, they be Clojure vectors, lists, or hash-maps.

The higher-level case is to store reversing information in hash-maps
along with base-data. The base data will belong to the \emph{:data} key, by
convention.


\begin{mydefinition}[Reversible Algebraic Vector]
A \textbf{reversible algebraic vector} is either a \textbf{base-data}
collection or a hash-map containing a \texttt{:data} attribute. A
base-data collection is either a Clojure vector, list, or hash-map that
does not contain a \texttt{:data} attribute. If a reversible algebraic
vector does contain a \texttt{:data} attribute, the value of that
attribute is a base-data collection.  \end{mydefinition}

Here is a \emph{fluent} type-checking function for base-data. It either
returns its input -- like the \emph{identity} function -- or throws an
exception if something is wrong.

\begin{figure}[H]
\label{check-data-type}
\begin{verbatim}
(defn- check-data-type [that]
  (let [t (type that)]
    (if (or (= t (type []))
            (= t (type '()))  ; empty list is special
            (= t (type '(0))) ; this list is ordinary
            (and (= t (type {})) (not (contains? that :data))))
      that ; ok -- otherwose:
      (throw (IllegalArgumentException.
        (str "This type of object can't hold vector data: " t))))))
\end{verbatim}
\end{figure}

Now we need a way to get the data out of any reversible algebraic
vector.

If the input is a hash-map, we must explicitly check for existence of
\emph{:data} so that we can tell the difference between a hash-map that has
\emph{:data} whose value is \emph{nil}, which is an illegal case, and a hash-map
that has no \emph{:data}, a legal case. We cannot simply apply the keyword
\emph{:data} to the candidate reversible vector because that application
would produce \emph{nil} in both cases. Instead, we apply \emph{:data} to the
candidate after checking for existence of the key, and then we apply
\emph{check-data-type}, defined above.

\begin{figure}[H]
\label{get-data-helper}
\begin{verbatim}
(defmulti  get-data type)
(defmethod get-data (type [])   [that] that)
(defmethod get-data (type '())  [that] that)
(defmethod get-data (type '(0)) [that] that)
(defmethod get-data (type {})   [that]
  (if (contains? that :data)
    (check-data-type (:data that))
    that))
(defmethod get-data :default    [that]
  (throw (IllegalArgumentException.
    (str "get-data doesn't like this food: " that))))
\end{verbatim}
\end{figure}

Now we write a test for all these cases. We require
\emph{IllegalArgumentExceptions} for bases-data blocks that are not vectors,
lists, or hash-maps or base-data blocks that contain
reversible-vectors: our design does not want to nest such vectors.

\begin{figure}[H]
\label{test-get-data-helper}
\begin{verbatim}
(deftest get-data-helper-test
  (testing "get-data-helper"
    ;; Negative tests
    (are [val] (thrown? IllegalArgumentException val)
      (get-data 42)
      (get-data 'a)
      (get-data :a)
      (get-data "a")
      (get-data \a)
      (get-data #inst "2012Z")
      (get-data #{})
      (get-data nil)
      (get-data {:data 42 })
      (get-data {:data 'a })
      (get-data {:data :a })
      (get-data {:data "a"})
      (get-data {:data \a })
      (get-data {:data #inst "2012Z"})
      (get-data {:data #{} })
      (get-data {:data nil })
      (get-data {:data {:data 'foo} })
    )
    ;; Positive tests
    (are [x y] (= x y)
          [] (get-data  [])
         '() (get-data '())
          {} (get-data  {})

         [0] (get-data  [0])
        '(0) (get-data '(0))
      {:a 0} (get-data  {:a 0})

       [1 0] (get-data  [1 0])
      '(1 0) (get-data '(1 0))
 {:a 0 :b 1} (get-data  {:b 1 :a 0})

        [42] (get-data {:a 1 :data [42]})
       '(42) (get-data {:a 1 :data '(42)})
     {:a 42} (get-data {:a 1 :data {:a 42}})

          [] (get-data {:a 1 :data []})
         '() (get-data {:a 1 :data '()})
          {} (get-data {:a 1 :data {}})
    )
))
\end{verbatim}
\end{figure}


To implement the protocol, we will need multimethods that dispatch on
the types of the base data. There is an example of this above in get
data; let's follow it to build add-data:

\begin{figure}[H]
\label{add-data}
\begin{verbatim}
(defn two-types [a b]
(defmulti  add-data two-types)
(defmethod add-data (type [])   [that] that)
(defmethod add-data (type '())  [that] that)
(defmethod add-data (type '(0)) [that] that)
(defmethod add-data (type {})   [that]
  (if (contains? that :data)
    (check-data-type (:data that))
    that))
(defmethod add-data :default    [that]
  (throw (IllegalArgumentException.
    (str "get-data doesn't like this food: " that))))
\end{verbatim}
\end{figure}


\begin{figure}[H]
\label{reversible-algebraic-vector-on-vector}
\begin{verbatim}
(defrecord ReversibleVector [a-vector]
  IReversibleAlgebraicVector
  (add   [a b] {:left-prior a, :right-prior b,
                :operation 'add, :data (map + (get-data a)
                                              (get-data b))})
  (sub   [a b] nil)
  (inner [a b] nil)
  (scale [a scalar] nil)
  (undo  [a] nil)
  (redo  [b] nil))
\end{verbatim}
\end{figure}
\section{Unit-Tests}
\label{sec-4}

\begin{figure}[H]
\label{test-namespace}
\begin{verbatim}
(ns ex1.core-test
  (:require [clojure.test :refer :all]
            [ex1.core     :refer :all]))
\end{verbatim}
\end{figure}
\section{REPLing}
\label{sec-5}
\label{sec:emacs-repl}
To run the REPL for interactive programming and testing in org-mode,
take the following steps:
\begin{enumerate}
\item Set up emacs and nRepl (TODO: explain; automate)
\item Edit your init.el file as follows (TODO: details)
\item Start nRepl while visiting the actual |project-clj| file.
\item Run code in the org-mode buffer with \verb|C-c C-c|; results of
evaluation are placed right in the buffer for inspection; they are
not copied out to the PDF file.
\end{enumerate}
% Emacs 24.3.1 (Org mode 8.0)
\end{document}
